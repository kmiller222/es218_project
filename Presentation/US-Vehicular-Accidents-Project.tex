\PassOptionsToPackage{unicode=true}{hyperref} % options for packages loaded elsewhere
\PassOptionsToPackage{hyphens}{url}
%
\documentclass[]{article}
\usepackage{lmodern}
\usepackage{amssymb,amsmath}
\usepackage{ifxetex,ifluatex}
\usepackage{fixltx2e} % provides \textsubscript
\ifnum 0\ifxetex 1\fi\ifluatex 1\fi=0 % if pdftex
  \usepackage[T1]{fontenc}
  \usepackage[utf8]{inputenc}
  \usepackage{textcomp} % provides euro and other symbols
\else % if luatex or xelatex
  \usepackage{unicode-math}
  \defaultfontfeatures{Ligatures=TeX,Scale=MatchLowercase}
\fi
% use upquote if available, for straight quotes in verbatim environments
\IfFileExists{upquote.sty}{\usepackage{upquote}}{}
% use microtype if available
\IfFileExists{microtype.sty}{%
\usepackage[]{microtype}
\UseMicrotypeSet[protrusion]{basicmath} % disable protrusion for tt fonts
}{}
\IfFileExists{parskip.sty}{%
\usepackage{parskip}
}{% else
\setlength{\parindent}{0pt}
\setlength{\parskip}{6pt plus 2pt minus 1pt}
}
\usepackage{hyperref}
\hypersetup{
            pdftitle={US Vehicular Accidents},
            pdfauthor={Kirsten Miller},
            pdfborder={0 0 0},
            breaklinks=true}
\urlstyle{same}  % don't use monospace font for urls
\usepackage[margin=1in]{geometry}
\usepackage{graphicx,grffile}
\makeatletter
\def\maxwidth{\ifdim\Gin@nat@width>\linewidth\linewidth\else\Gin@nat@width\fi}
\def\maxheight{\ifdim\Gin@nat@height>\textheight\textheight\else\Gin@nat@height\fi}
\makeatother
% Scale images if necessary, so that they will not overflow the page
% margins by default, and it is still possible to overwrite the defaults
% using explicit options in \includegraphics[width, height, ...]{}
\setkeys{Gin}{width=\maxwidth,height=\maxheight,keepaspectratio}
\setlength{\emergencystretch}{3em}  % prevent overfull lines
\providecommand{\tightlist}{%
  \setlength{\itemsep}{0pt}\setlength{\parskip}{0pt}}
\setcounter{secnumdepth}{0}
% Redefines (sub)paragraphs to behave more like sections
\ifx\paragraph\undefined\else
\let\oldparagraph\paragraph
\renewcommand{\paragraph}[1]{\oldparagraph{#1}\mbox{}}
\fi
\ifx\subparagraph\undefined\else
\let\oldsubparagraph\subparagraph
\renewcommand{\subparagraph}[1]{\oldsubparagraph{#1}\mbox{}}
\fi

% set default figure placement to htbp
\makeatletter
\def\fps@figure{htbp}
\makeatother


\title{US Vehicular Accidents}
\author{Kirsten Miller}
\date{5/15/2020}

\begin{document}
\maketitle

{
\setcounter{tocdepth}{2}
\tableofcontents
}
\hypertarget{introduction}{%
\subsection{Introduction}\label{introduction}}

I investigated vehicular accidents in the United States over a 21-year
period (1996 to 2016). There are many potential ways to explore this
dataset, so I tried to understand it broadly while also focusing on some
more specific aspects and potential relationships. I first explored the
data broadly by visualizing the number of people involved in vehicular
accidents over varying timescales. I then looked at the breakdown of the
number of people involved in accidents by many of the variables included
to get a sense of the dataset. After this more general visualization, I
attempted to uncover some of the relationships between specific
variables and the number of people involved in accidents. In particular,
I focused on injury severity, age, and alcohol involvement. I also
focused on local states in portions of my analysis (Maine, New
Hampshire, Vermont, and Massachusetts).

Some questions that guided this data exploration include:

\begin{itemize}
\tightlist
\item
  How has the number of vehicular accidents changed over different
  timescales (years, months, hours)?
\item
  What is the breakdown of types of injury severity and has this changed
  over time?
\item
  Are there significant differences between the ages of people involved
  in certain types of accidents?
\end{itemize}

While answering these questions, I also considered:

\begin{itemize}
\tightlist
\item
  How do vehicular accidents vary among regions of the country and local
  states and how does this compare to the United States as a whole?
\item
  Could involvement of alcohol be a predictor for certain types of
  accidents?
\end{itemize}

\hypertarget{methods}{%
\subsection{Methods}\label{methods}}

This dataset was collected by the National Highway Traffic Safety
Administration (NHTSA) for the years 1996 through 2016. The data
includes the following information for each person involved in a
vehicular accident: state, county, month, day, year, hour, minute,
manner of collision, number of vehicles involved, type of vehicle
involved, number of people involved, age of driver, sex of driver,
involvement of alcohol, and severity of injury.

An important note is that the data are broken down by individuals
involved in vehicular accidents, not the unique accidents themselves.
Therefore, my analysis is based on the number of individuals involved in
vehicular accidents, not the number of vehicular accidents.

\hypertarget{results}{%
\subsection{Results}\label{results}}

\hypertarget{number-of-people-involved-in-vehicular-accidents}{%
\subsubsection{Number of People Involved In Vehicular
Accidents}\label{number-of-people-involved-in-vehicular-accidents}}

\hypertarget{number-of-people-involved-in-vehicular-accidents-in-the-us-over-a-21-year-period}{%
\paragraph{Number of people involved in vehicular accidents in the US
over a 21-year
period}\label{number-of-people-involved-in-vehicular-accidents-in-the-us-over-a-21-year-period}}

\includegraphics{US-Vehicular-Accidents-Project_files/figure-latex/unnamed-chunk-3-1.pdf}

First, I examined the number of people involved in vehicular accidents
in the US over the entire 21 year time period for the dataset (1996 -
2016). I used a loess fit and found that a second-degree loess fit was
the best fit for this plot. The number of people involved in vehicular
accidents was fairly constant (around 100,000 accidents) over the first
10 years of the dataset from 1996 to 2005. This was followed by a
decrease in the number of people involved in accidents to a low number
of 73073 in 2011, followed by a recent increase to 85496 accidents in
2016. The roughly linear trend prior to 2005 followed by a decrease in
the number of people involved in accidents suggests that there could
have been either a change in the method of data reporting or a change in
regulations around 2005; however I was not able to find any concrete
information to confirm or deny this.

\hypertarget{analytical-plots-for-residuals}{%
\subparagraph{Analytical plots for
residuals}\label{analytical-plots-for-residuals}}

A second-degree loess was the best fit for the overall plot. The
residual-dependence plot shows that the loess fit does approximate a
horizontal line; however, there appears to be a potential fanning
pattern of the residuals (the points become more scattered when moving
from left to right) seen in the residual-dependence plot. The
spread-location plot shows a decreasing spread of the original data,
indicating that the variability decreases. I then checked the residuals
for normality by comparing to the normal distribution (residuals
vs.~theoretical plot). The residuals align somewhat well with a
theoretic distribution, although there may be some level of skew to the
left.

\includegraphics[width=0.333\linewidth]{US-Vehicular-Accidents-Project_files/figure-latex/unnamed-chunk-4-1}
\includegraphics[width=0.333\linewidth]{US-Vehicular-Accidents-Project_files/figure-latex/unnamed-chunk-4-2}
\includegraphics[width=0.333\linewidth]{US-Vehicular-Accidents-Project_files/figure-latex/unnamed-chunk-4-3}

\hypertarget{number-of-people-involved-in-accidents-over-time-in-local-states}{%
\paragraph{Number of people involved in accidents over time in local
states}\label{number-of-people-involved-in-accidents-over-time-in-local-states}}

\includegraphics{US-Vehicular-Accidents-Project_files/figure-latex/unnamed-chunk-5-1.pdf}

The number of people involved in accidents over time in local states
appears to follow a similar pattern to the overall trend for the United
States. Again, there is a change after 2005 (most pronouced in
Massachusetts) which could indicate a change in the method of data
reporting. In addition, with smaller total numbers of people involved in
accidents in these states, the year-to-year variability is more apparent
(compared to the number of people involved in accidents in the US as a
whole). This also indicates the varying number of people involved in
accidents by state, which is examined further below:

\hypertarget{cumulative-number-of-people-involved-in-accidents-in-each-state-from-1996---2016}{%
\paragraph{Cumulative number of people involved in accidents in each
state from 1996 -
2016}\label{cumulative-number-of-people-involved-in-accidents-in-each-state-from-1996---2016}}

\includegraphics{US-Vehicular-Accidents-Project_files/figure-latex/unnamed-chunk-6-1.pdf}

The number of people involved in accidents in each state varies widely.
The top 5 states with the greatest cumulative number of people involved
in accidents over the 21-year period were California, Texas, Florida,
Georgia, and North Carolina, while the 5 states with the least number of
people involved in accidents were the District of Columbia, Rhode
Island, Vermont, Alaska and North Dakota.

To further examine how the number of people involved in accidents
changes over different timescales, I looked at the distribution of the
total number of people involved in accidents over each month of the
year.

\hypertarget{number-of-people-involved-in-accidents-per-month-of-the-year}{%
\paragraph{Number of people involved in accidents per month of the
year}\label{number-of-people-involved-in-accidents-per-month-of-the-year}}

\includegraphics{US-Vehicular-Accidents-Project_files/figure-latex/unnamed-chunk-7-1.pdf}

I was surprised to see that the cumulative number of people involved in
accidents was higher during the summer months (with the highest number
occurring in July and the lowest number occurring in February). I would
have predicted that there would be a greater number of accidents during
the winter because of more weather events and difficult driving
conditions. To examine this further, I looked at the number people
involved in accidents each month by region (assuming that northern
regions would experience more weather that could impact driving in the
winter).

\hypertarget{number-of-people-involved-in-accidents-per-month-by-region}{%
\paragraph{Number of people involved in accidents per month by
region}\label{number-of-people-involved-in-accidents-per-month-by-region}}

\includegraphics{US-Vehicular-Accidents-Project_files/figure-latex/unnamed-chunk-8-1.pdf}

Here, I was again surprised to see that the number of people involved in
accidents was still higher in the summer in both the Northeast and North
Central regions. However, the greater number people involved in
accidents during the summer could be due to increased travel during
these months. It was interesting to see that many more people were
involved in accidents overall in the South Region (although this is
likely because this region has a higher population).

Next, I looked at the distribution of the total number of people
involved in accidents per hour of the day:

\hypertarget{number-of-people-involved-in-accidents-per-hour-of-the-day}{%
\paragraph{Number of people involved in accidents per hour of the
day}\label{number-of-people-involved-in-accidents-per-hour-of-the-day}}

\includegraphics{US-Vehicular-Accidents-Project_files/figure-latex/unnamed-chunk-9-1.pdf}

This shows the most people were involved in accidents that occurred
around 17:00, which is intuitive, as this is around the time when many
people return home from work, as well as when it may be getting dark. It
is also interesting to note that there is a small peak in the observed
distribution at 6:00-7:00, which might be explained by the morning
commute.

I was interested to see how alcohol involvement might be related to this
distribution:

\hypertarget{number-of-people-involved-in-accidents-during-each-hour-of-the-day-by-alcohol-involvement}{%
\paragraph{Number of people involved in accidents during each hour of
the day, by alcohol
involvement}\label{number-of-people-involved-in-accidents-during-each-hour-of-the-day-by-alcohol-involvement}}

\includegraphics{US-Vehicular-Accidents-Project_files/figure-latex/unnamed-chunk-10-1.pdf}

When alcohol was not involved, not reported, or unknown, the
distribution of the number of people involved in accidents per hour is
similar to the overall distribution. However, when alcohol is involved,
the distribution is near opposite, with the peak of accidents occurring
during the nighttime hours.

\hypertarget{number-of-people-involved-in-vehicular-accidents-by-day-of-the-week}{%
\paragraph{Number of people involved in vehicular accidents by day of
the
week}\label{number-of-people-involved-in-vehicular-accidents-by-day-of-the-week}}

\includegraphics{US-Vehicular-Accidents-Project_files/figure-latex/unnamed-chunk-11-1.pdf}

Finally, I examined the number of people involved in vehicular accidents
by day of the week. The most people were involved in accidents that
occurred on Friday and the weekend, while less people were involved in
accidents that occurred during the week. The greatest number of people
were involved in accidents on Saturdays (356068) whereas the least
number of people were involved in accidents that occurred on Tuesdays
(221969).

\hypertarget{number-of-people-involved-in-vehicular-accidents-by-day-of-the-week-by-alcohol-involvement}{%
\paragraph{Number of people involved in vehicular accidents by day of
the week, by alcohol
involvement:}\label{number-of-people-involved-in-vehicular-accidents-by-day-of-the-week-by-alcohol-involvement}}

\includegraphics{US-Vehicular-Accidents-Project_files/figure-latex/unnamed-chunk-12-1.pdf}

Again, I also looked at whether this breakdown by day of the week
changed at all with alcohol involvement. When alcohol was involved
(``Yes''), the distribution appears to be similar to the overall
distribution, although the number of people involved in accidents on the
weekend appears to be proportionally higher. When alcohol was not
involved (``No''), the number of people involved in accidents is fairly
consistent from Sunday through Thursday, and slightly larger on Friday
and Saturday.

\hypertarget{severity-of-injury-sustained}{%
\subsubsection{Severity of Injury
Sustained}\label{severity-of-injury-sustained}}

After an initial investigation of some more of the dataset variables,
(including vehicle type, manner of collision, and number of people
involved), I decided that I was most interested in looking at injury
severity. Below is the overall breakdown of injury severity for the
cumulative number of people involved in accidents:

\hypertarget{injury-severity}{%
\paragraph{Injury Severity}\label{injury-severity}}

\includegraphics{US-Vehicular-Accidents-Project_files/figure-latex/unnamed-chunk-13-1.pdf}

Out of all the the people involved in accidents, fatal injuries makes up
the largest category of injury level, with 815926 total fatalities
recorded in this dataset. This is over twice as many as those who were
reported to have no apparent injury. Those with possible, minor, or
serious injuries also made up significant portions of the total number
of people involved in accidents.

I also was curious to see if there were any evident differences in
injury severity in local states, shown below:

\hypertarget{severity-of-injury-by-state}{%
\paragraph{Severity of injury by
state}\label{severity-of-injury-by-state}}

\includegraphics{US-Vehicular-Accidents-Project_files/figure-latex/unnamed-chunk-14-1.pdf}

The breakdown of the categories of injury severity is fairly consistent
across these states, with fatal injuries between 46 and 48 percent for
all four states. The proportions are also fairly consistent with the
overall breakdown for the United States above.

Next, I looked at injury severity by alcohol involvement:

\hypertarget{severity-of-injury-by-alcohol-involvment}{%
\paragraph{Severity of injury by alcohol
involvment}\label{severity-of-injury-by-alcohol-involvment}}

\includegraphics{US-Vehicular-Accidents-Project_files/figure-latex/unnamed-chunk-15-1.pdf}

When comparing the accidents in which it was known whether alcohol was
involved (``Yes'') or not involved (``No''), it appears the proportion
of fatal injuries was much higher when alcohol was involved. Fatal
injuries made up 64.7\% of accidents involving alcohol, while they only
accounted for 36.6\% of those accidents not involving alcohol.

I also looked at how the different types of injury had changed over
time:

\hypertarget{severity-of-injury-over-time}{%
\paragraph{Severity of injury over
time}\label{severity-of-injury-over-time}}

\includegraphics{US-Vehicular-Accidents-Project_files/figure-latex/unnamed-chunk-16-1.pdf}

All types of injuries appear to have decreased over time, although after
more substantial decreases between 2005 and 2010, injuries (as well as
the no injury category) have increased again in the last 5 years of the
dataset. Recalling the trend for overall number of people involved in
accidents (the first figure in the results section), this decrease
followed by a recent increase is very similar.

\hypertarget{severity-of-injury-over-time-in-local-states}{%
\paragraph{Severity of injury over time in local
states}\label{severity-of-injury-over-time-in-local-states}}

\includegraphics{US-Vehicular-Accidents-Project_files/figure-latex/unnamed-chunk-17-1.pdf}

Again, the patterns over time in local states were relatively consistent
and similar to the overall trend for the United States, with more
year-to-year variability.

\hypertarget{age-of-people-involved-in-vehicular-accidents}{%
\subsubsection{Age of People Involved In Vehicular
Accidents}\label{age-of-people-involved-in-vehicular-accidents}}

I was also interested in investigating the distribution of the age of
people involved in vehicular accidents. I first looked at the
distribution of the ages of people involved in vehicular accidents, and
whether their average age has changed over time.

\hypertarget{density-plot-for-age-of-people-involved-in-vehicular-accidents}{%
\paragraph{Density plot for age of people involved in vehicular
accidents}\label{density-plot-for-age-of-people-involved-in-vehicular-accidents}}

\includegraphics{US-Vehicular-Accidents-Project_files/figure-latex/unnamed-chunk-18-1.pdf}

This density plot shows the distribution of the age of people involved
in vehicular accidents. The median age of people involved in vehicular
accidents over this time period was 33. However, looking at the density
plot for age, the peak density occurs in the low 20's. This is followed
by a somewhat consistent density level from age 25 to age 50, followed
by a decrease in density as age continues to increase.

\hypertarget{number-of-people-involved-in-accidents-for-each-state-and-mean-age}{%
\paragraph{Number of people involved in accidents for each state and
mean
age}\label{number-of-people-involved-in-accidents-for-each-state-and-mean-age}}

\includegraphics{US-Vehicular-Accidents-Project_files/figure-latex/unnamed-chunk-19-1.pdf}

The plot above displays the number of people involved in accidents in
each state, as well as the mean age in each state. The mean age of
people involved in accidents does varies noticeably among the states. To
further investigate this, I created a scatterplot comparing mean age and
total number of accidents by state:

\includegraphics{US-Vehicular-Accidents-Project_files/figure-latex/unnamed-chunk-20-1.pdf}

There is not much of a trend, and the three states with the highest
number of people involved in accidents (California, Texas, and Florida)
are noticeably separated from the rest of the data points. Based on this
scatterplot, there does not appear to be a strong relationship between
number of people involved in accidents and mean age in each state. The
variation in mean age and number of people involved in acccidents
between states could be influenced by the mean age and the total
population overall in each state, so to further investigate it would be
best to normalize both values based on the overall state mean age and
population.

Next, I examined whether there has been a change in the age of people
involved in vehicular accidents over time by plotting the mean age by
year:

\hypertarget{mean-age-of-people-involved-in-vehicular-accidents-over-time}{%
\paragraph{Mean Age of People Involved In Vehicular Accidents Over
Time}\label{mean-age-of-people-involved-in-vehicular-accidents-over-time}}

\includegraphics{US-Vehicular-Accidents-Project_files/figure-latex/unnamed-chunk-22-1.pdf}

The mean age of people involved in vehicular accidents increased over
time from 34.6877506 in 1996 to 39.2625147 in 2016. Although there was
some inconsistency in the residuals, the data was best approximated by
second degree polynomial fit, defining the relationship in the following
equation:

\(Mean Age = 0.007(Year)^2 - 26.48(Year) + 26355.23\)

\includegraphics{US-Vehicular-Accidents-Project_files/figure-latex/unnamed-chunk-23-1.pdf}

\hypertarget{age-by-group-alcohol-involvement-injury-severity}{%
\subsubsection{Age By Group: Alcohol Involvement \& Injury
Severity}\label{age-by-group-alcohol-involvement-injury-severity}}

\hypertarget{age-of-people-involved-in-accidents-by-alcohol-involvement}{%
\paragraph{Age of people involved in accidents, by alcohol
involvement}\label{age-of-people-involved-in-accidents-by-alcohol-involvement}}

\includegraphics{US-Vehicular-Accidents-Project_files/figure-latex/unnamed-chunk-24-1.pdf}

The median age of those involved in vehicle accidents when alcohol was
involved was 32, and when alcohol was not involved the median age was
38. This suggests that the age of people involved in vehicle accidents
with alcohol involvement could be lower than the age of people involved
in accidents without alcohol involvement. Ideally, one would perform a
parametric test to determine whether the difference in age between the
two groups was significant or not. Below, I check the conditions
required for performing a t-test. Unfortunately, the distributions are
not normal and do not match each other so it is not possible to perform
a t-test. After the peak around age 20 for both groups (the
distributions are skewed to the right), there is a hump in the
distribution at around age 45 which is even more pronounced for the
alcohol not involved group. The alcohol-involved group also does not
include many individuals under age 20, since people do not usually drive
with their kids when drinking is involved, but this age group is
represented in the alcohol-not-involved group.

\includegraphics{US-Vehicular-Accidents-Project_files/figure-latex/unnamed-chunk-26-1.pdf}

Next, I investigated whether there was a significant difference between
the age of people involved in accidents who experienced fatal injuries
and those who experienced nonfatal injuries.

\hypertarget{age-of-people-who-suffered-fatal-injuries-versus-non-fatalno-injuries}{%
\paragraph{Age of people who suffered fatal injuries versus non-fatal/no
injuries}\label{age-of-people-who-suffered-fatal-injuries-versus-non-fatalno-injuries}}

\includegraphics{US-Vehicular-Accidents-Project_files/figure-latex/unnamed-chunk-27-1.pdf}

The median age of those who suffered fatal injuries was 38, compared to
30 for those who did not experience a fatal injury. This suggests that
the age of those who experience a fatal injury may be older than those
who experience a non-fatal or no injury. To test whether this difference
is significant, a t-test could be performed. However, the conditions for
a t-test are not met in this case because the distributions of both
groups do not match the normal distribution (seen in the density plots
below). Both groups are skewed to the right. In the group that
experienced fatal injury, the density is higher in the middle age
groups, with humps around 40 and 80, which are not as pronounced in the
non-fatal group, which has a greater density for younger ages (below age
20).

\includegraphics{US-Vehicular-Accidents-Project_files/figure-latex/unnamed-chunk-29-1.pdf}

\hypertarget{quantile---quantile-plot}{%
\subparagraph{Quantile - Quantile plot}\label{quantile---quantile-plot}}

The q-q plots below compare the values of the fatal group to those of
the non-fatal group. They show that there is a systematic difference
between the age of people who experienced fatal injuries and those who
did not, with the age of those who experienced fatal injury being
higher. In the plot on the right, this offset is quantified by adding 7
to the age of the non-fatal group. However, this q-q plot also reveals
that the data should be divided into multiple groups for further
analysis because the fit is not completely linear. The greatest
discrepancy can be seen for below age 20, where the trend is most
noticeably different from the line. There is also a lot of variability
above age 90. Therefore, it would be best to determine the relationship
between injury severity and age by dividing the data into two separate
age groups: 0-20 and 20-90, because the relationship will likely be
different for each group.

\includegraphics[width=0.5\linewidth]{US-Vehicular-Accidents-Project_files/figure-latex/unnamed-chunk-30-1}
\includegraphics[width=0.5\linewidth]{US-Vehicular-Accidents-Project_files/figure-latex/unnamed-chunk-30-2}

\hypertarget{discussion}{%
\subsection{Discussion}\label{discussion}}

This exploratory analysis investigated various aspects of vehicular
accidents in the United States over a 21-year period. The main areas of
focus were change over varying timescales, injury severity, age of
people involved, and alcohol involvement.

By analyzing this dataset, I found that the number of people involved in
vehicle accidents, as well as the number of fatalities (in the US as a
whole and in local states), has decreased over time. There are
indications of a potential systematic decrease around 2005, although I
did not find evidence to confirm this. However, both number of people
involved in accidents and fatalities have increased in the last few
years of the data examined. The number of accidents also varies by month
of the year, hour of the day, and day of the week; accidents are higher
during the summer months, evening hours, and weekend days.

Fatalities make up a significant component of the people who experienced
vehicular accidents recorded in this dataset, and proportions of fatal
injury as well as change over time of fatal injury are relatively
consistent in local states and close to 50\% for this dataset.

The mean age of people involved in vehicular accidents has increased
over time. The age of people who experience fatal injuries through
vehicular accidents is typically higher than the age of those who do
not.

Accidents involving alcohol are especially prevalent during nighttime
hours and on weekend days. The proportion of fatal injuries is higher
for people in accidents for whom alcohol was involved. These data also
suggest that alcohol involvement is more common in younger people who
are involved in accidents.

These results identify certain time periods during which more people are
usually involved in accidents, which is helpful for understanding risk,
as well as in consideration of how vehicle accidents can be managed and
prevented. These findings also suggest that both young people and older
people may be at risk for vehicle accidents due to differing reasons:
young people may have more accidents related to alcohol use, while older
people may be more at risk for fatal injuries if they are to be involved
in a vehicular accident.

There are many ways that this dataset could be explored more to
highlight these potential relationships, as well as to uncover further
trends and patterns. Perhaps most pertinently, it will be important to
understand why the number of people involved in accidents has again
increased over the last 5 years of this dataset in order to reduce this
number in the future. To further this analysis, I would suggest
examining the trends over time while normalizing the data for overall
population trends to see if some of the findings could be influenced by
the characteristics of the general population. I would also suggest that
more studies consider alcohol involvement in vehicle accidents, since
this analysis found that it may be connected to fatalities as well as
accidents involving young people.

\hypertarget{references}{%
\subsection{References}\label{references}}

NHTSA Data: \url{https://www.nhtsa.gov}

R Core Team (2019). R: A language and environment for statistical
computing. R Foundation for Statistical Computing, Vienna, Austria. URL
\url{https://www.R-project.org/}.

Gimond, M. (2020). Exploratory Data Analysis in R.
\url{https://mgimond.github.io/ES218/index.html}

\end{document}
